\section{Interpolation}

    The main reference of this section is \cite{Amerik11nonpreperiodic,BGT10DMLetale,Poonen14interpolation,Xie25ZDOsurfaces}.
    
    % Let \(\kk\) be a finitely generated field over \(\bbQ\) and \(\kkk\) be its algebraic closure.
    % Let \(X\) be a variety over \(\kk\) and \(f: X \ratmap X\) a dominant rational self-map defined over \(\kk\).
    % We want to show that after iteration, we can interpolate the iterates of \(f\) on an analytic open subset of \(X(\bbC_p)\) for some prime \(p\).

\subsection{Interpolation of analytic maps}

    In this subsection, we find the interpolation of analytic maps on an analytic disk.
    Fix a complete non-archimedean field \(\kk\) of characteristic \(0\) with \(|p|_{\kk} = 1/p\) for some prime \(p\).
    Set \(r_p = p^{-1/(p-1)}\).
    We use the method of difference operators given in \cite{Poonen14interpolation}.

    Let \(E = E(0,1) = \{x \in \kk^d \mid \|x\| \leq 1\}\) be the closed unit ball in \(\kk^d\) and \(\Phi: E \to E\) be an analytic map, i.e., \(\Phi \in \kk^\circ\{ \underline{X} \}^d\).
    Here the norm on \(\kk^d\) or \(\kk^\circ\{ \underline{X} \}^d\) is the supremum norm, i.e., \(|x| = \max_{1 \leq i \leq d} |x_i|\).
    For every analytic map \(h\) from \(E\) to \(E\), we define 
    \[ \Delta(h) := h \circ \Phi - h, \quad \Delta^n(h) := \Delta(\Delta^{n-1}(h)) \text{ for } n \geq 1, \]
    and \(\Delta^0(h) = h\).
    Note that \(\Delta^n(h)\) is still an analytic map from \(E\) to \(E\) by the strong triangle inequality.

    \begin{lemma}\label{lem:difference_operators_and_binomial_theorem}
        We have the following binomial theorem:
        \[ \sum_{m=0}^n \binom{n}{m} \Delta^m(\id_E) = \Phi^n. \]
    \end{lemma}
    \begin{proof}
        By induction, we have 
        \begin{align*}
            \Delta^m(\id_E) &= \Delta\left( \sum_{k=0}^{m-1} \binom{m-1}{k} (-1)^{m-1-k} \Phi^k \right) \\
                &= \sum_{k=0}^{m-1} \binom{m-1}{k} (-1)^{m-1-k} \Phi^{k+1} - \sum_{k=0}^{m-1} \binom{m-1}{k} (-1)^{m-1-k} \Phi^k \\
                &= \sum_{k=0}^{m} \left( \binom{m-1}{k-1} (-1)^{m-k} - \binom{m-1}{k} (-1)^{m-1-k} \right) \Phi^k \\
                &= \sum_{k=0}^{m} \binom{m}{k} (-1)^{m-k} \Phi^k.
        \end{align*}
        It follows that 
        \begin{align*}
            \sum_{m=0}^n \binom{n}{m} \Delta^m(\id_E) &= \sum_{m=0}^n \sum_{k=0}^m \binom{n}{m} \binom{m}{k} (-1)^{m-k} \Phi^k \\
                &= \sum_{k=0}^n\sum_{m=k}^n \binom{n}{k} \binom{n-k}{m-k} (-1)^{m-k} \Phi^k  \\
                &= \sum_{k=0}^n \binom{n}{k} \Phi^k  \sum_{l=0}^{n-k} \binom{n-k}{l} (-1)^{l} \\
                &= \binom{n}{n} \Phi^n + \sum_{k=0}^{n-1} \binom{n}{k} \Phi^k \cdot (1-1)^{n-k} \\
                &= \Phi^n.
        \end{align*}
        We finish the proof.
    \end{proof}

    \begin{lemma}\label{lem:norm_of_difference_operators_of_Phi}
        Suppose that \(\Phi = (\Phi_1,\cdots,\Phi_d)\in \kk^\circ\{ \underline{X} \}^d\) satisfies \(\|\Phi - \id_E\| \leq r\).
        Then 
        \[ \|\Delta^n(\id_E)\| \leq r^n \]
        % \Yang{To be added.}
    \end{lemma}
    \begin{proof}
        By \Yang{ref}, we have 
        \[ \|\Delta(h)\| = \|h \circ \Phi - h\| \leq \|h\| \cdot \|\Phi - \id_E\| \leq r \|h\|. \]
        Hence by induction, the result follows.
        % \Yang{To be added.}
    \end{proof}

    \begin{theorem}[{ref.\cite[Theorem 1]{Poonen14interpolation},\ cf.\cite[Theorem 3.3]{BGT10DMLetale}}]\label{prop:Poonen14_p-adic_interpolation_of_iterates}
        Suppose that \(\Phi = (\Phi_1,\cdots,\Phi_d)\in \kk^\circ\{ \underline{X} \}^d\) satisfies \(r \coloneqq \|\Phi - \id_E\| < r_p\).
        Then there exists a unique function \(F \in \kk\{ \underline{X}, T/s \}^d\), such that for each \(n \in \bbZ_{\geq 0}\) and each \(x \in E\),
        \[F(x, n) = \Phi^n(x).\]
        Here \(s\) is any real number with \(1 < s < r_p/r\).
    \end{theorem}
    \begin{proof}
        Consider the formal series
        \[ F(\underline{X}, T) \coloneqq \sum_{n=0}^\infty \binom{T}{n} \Delta^n(\id_E)(\underline{X}). \]
        Recall the Newton's binomial function
        \[ \binom{T}{n} \coloneqq \frac{T(T-1)\cdots(T-n+1)}{n!} \in \kk[T]. \]
        Since \(\binom{T}{n}\) is a polynomial in \(T\) and \(\Delta^n(\id_E)(\underline{X}) \in \kk^\circ\{\underline{X}\}^d\), 
        we have \(f_n = \binom{T}{n} \Delta^n(\id_E)(\underline{X}) \in \kk\{\underline{X}, T\}^d\).
        Note that for each \(n \in \bbZ_{\geq 0}\), then \(|n!|_{\kk} \geq r_p^n\).
        Hence we have 
        \[ \left\|\binom{T}{n}\right\| \leq s^n r_p^{-n} \]
        since \(s > 1\).
        By \cref{lem:norm_of_difference_operators_of_Phi}, we have
        \[ \|f_n\| \leq \left\| \binom{T}{n} \right\| \cdot \|\Delta^n(\id_E)\| \leq s^n r_p^{-n} r^n = (s r / r_p)^n. \]
        Since \(s r / r_p < 1\), the series \(F(\underline{X}, T) = \sum_{n=0}^\infty f_n\) converges in \(\kk\{\underline{X}, T/s\}^d\).
        By \cref{lem:difference_operators_and_binomial_theorem}, we have \(F(x, n) = \Phi^n(x)\) for each \(n \in \bbZ_{\geq 0}\) and each \(x \in E\).
        The uniqueness of \(F\) follows from \Yang{ref}.
    \end{proof}

    \Yang{If \(f\) is invertible, can we see that \(g\) is unique?}
    \Yang{It seems right.}

    \begin{example}
        Let \(\kk = \bbQ_p\) with \(p \geq 3\), and let \(\Phi \colon E \to E\) be the analytic map defined by \(\Phi(x) = px^2+x\).
        Then we have \(\|\Phi - \id_E\| = \|pT^2\| = 1/p < r_p\).
        \Yang{To be checked.}
    \end{example}


\subsection{Interpolation on an analytic open subset of morphisms}
% \subsection{Pick integral models}

    Let \(f: X \ratmap X\) be a dominant rational self-map of a projective variety of dimension \(d\) defined over a finitely generated field \(\kk\) over \(\bbQ\).
    We try to find some analytic local interpolation of the iterates of \(f\) on \(X(\bbC_p)\) for some prime \(p\).

    \begin{lemma}\label{lem:existence_of_good_model_for_smooth_things}
        There exists a subring \(R \subseteq \kk\) of finite type over \(\bbZ\), 
        a projective scheme \(\calX\) over \(\Spec R\) with generic fiber \(X\), 
        and a rational self-map \(\calf: \calX \ratmap \calX\) over \(\Spec R\) with generic fiber \(f\) 
        such that 
        \begin{enumerate}
            \item for every prime ideal \(\frakp\) of \(R\), the special fiber \(\calX_{\frakp}\) is geometrically integral and of the same dimension as \(X\);
            \item the union of non-smooth locus of \(\calX\) and indeterminacy locus, non-\'etale locus of \(\calf\) does not contain any entire special fiber \(\calX_{\frakp}\);
        \end{enumerate}

        Moreover, if \(X\) is smooth (resp.\ \(f\) is a morphism, resp.\ \(f\) is \'etale), then we can further require that \(\calX\) is smooth over \(\Spec R\) (resp.\ \(\calf\) is a morphism, resp.\ \(\calf\) is \'etale over \(\Spec R\)).
    \end{lemma}


    \Yang{We can embedd \(R\) into \(\bbC_p\) for some \(p\).}

    % The main reference of this section is \cite[Section 3.2]{Xie25ZDOsurfaces}.
    % We first state the main theorem of this section.

    \begin{theorem}[{ref.\cite[Proposition 3.24]{Xie25ZDOsurfaces}}]\label{thm:Xie25_DML_for_adelic_general_points}
        % Let \(\kk\) be a finitely generated field over \(\bbQ\), \(X\) a projective variety defined over \(\kk\), and \(f \colon X \ratmap X\) a dominant rational self-map defined over \(\kk\).
        
        There exists an iteration \(g = f^m\) of \(f\), an embedding \(\kk \hookrightarrow \bbC_p\) for some prime \(p\geq 3\), 
        an analytic open subset \(U\cong (\bbC_p^{\circ})^d \subseteq X(\bbC_p)\) and an analytic map \(\Phi: \bbC_p^\circ \times U \to U\) such that
        \begin{enumerate}
            \item \(g\) is well-defined on \(U\), \(U\) is invariant under \(g\) and \(\|g|_U - \id_U\| < 1/p\);
            \item \(\Phi(n, x) = g^n(x)\) for each \(n \in \bbZ_{\geq 0}\) and each \(x \in U\);
        \end{enumerate}
    \end{theorem}


    \begin{example}\label{eg:interpolation_for_Jordan_block_of_elliptic_curves}
        Let \(X = E \times E\) with \(E\) an elliptic curve without complex multiplication defined over a number field \(\kk\), 
        and let \(f \colon X \to X\) be the endomorphism defined by \((a,b) \mapsto (a + b, b)\).
        \Yang{To be continued.}
    \end{example}