\section{Adelic topology}

    The main reference of this section is \cite{Xie25ZDOsurfaces}.
    Fix an algebraically closed field \(\kkk\) which is the algebraic closure of a finitely generated field over \(\bbQ\).
    Let \(X\) be variety over \(\kkk\).

    In this section, we allow that the variety \(X\) is not irreducible, 
    i.e., a variety over \(\kkk\) is a reduced separated scheme of finite type over \(\kkk\).

    Let \(L/K\) be a field extension.
    We denote by \(M_{L/K}\) (resp. \(M_{L,K})\) the set of places of \(L\) which restrict to \(K\) is trivial (resp. non-trivial).


\subsection{Adelic topology}

    \begin{notation}\label{notation:embedding_set_for_adelic_topology}
        Set 
        \[ F(\kkk) \coloneqq \{\kk \subseteq \kkk \mid \kk \text{ is finitely generated over } \bbQ \text{ and } \kkk/\kk \text{ is algebraic}\}. \]
        Denote by \(I_\kk\) the set of all embeddings \(\sigma: \kk \to \bbC_\sigma\) over \(\kk\), where \(\bbC_\sigma = \bbC_p\) or \(\bbC\).
        Every such \(\sigma\) corresponds to a place \(v\in M_{\kk,\bbQ}\) by pulling back the standard absolute value on \(\bbC_p\) or \(\bbC\).
        For each \(\sigma\in I_\kk\), set \(E_\sigma = \{\bar{\sigma}: \kkk \to \bbC_\sigma \mid \bar{\sigma}|_\kk = \sigma\}\) be the set of embeddings of \(\kkk\) to \(\bbC_v\) extending \(\sigma\).
        
        We say that \(\kk\) is a \emph{defined field of \(X\)} if there exists a \(\kk\)-variety \(X_\kk\) such that \(X_\kk \times_{\kk} \Spec \kkk \cong X\).
        Such a variety \(X_\kk\) is called a \emph{model of \(X\) over \(\kk\)} or a \emph{\(\kk\)-model of \(X\)}.
        For every \(\sigma \in I_\kk\) and \(\bar{\sigma} \in E_\sigma\), we have an inclusion map \(\phi_{\bar{\sigma}}: X(\kkk) \cong X_{\kk}(\kkk) \to X_{\kk}(\bbC_{\sigma})\).
        Here the first isomorphism is given by the base change \(X_\kk \times_{\kk} \Spec \kkk \cong X\) and the second map is given by composition with \(\Spec \bar{\sigma}: \Spec \bbC_\sigma \to \Spec \kkk\).
        On \(X_{\kk}(\bbC_\sigma)\), we have the analytic topology induced from the norm on \(\bbC_\sigma\).
    \end{notation}

    
    \begin{definition}\label{def:basic_adelic_subset}
        Let \(\kk\) be a defined field of \(X\) with a model \(X_\kk\).
        Let \(\sigma,\sigma_i \in I_\kk\) and \(U \subset X_{\kk}(\bbC_\sigma), U_i \subseteq X_{\kk}(\bbC_{\sigma_i})\) be an open subsets in the analytic topology.
        We define
        \[ X_\kk(\sigma,U) \coloneqq \bigcup_{\tau\in E_\sigma} \phi_\tau^{-1}(U) \subseteq X(\kkk) \]
        and 
        \[ X_\kk(\{\sigma_i,U_i\}_{i=1}^n) \coloneqq \bigcap_{i=1}^n X_\kk(\sigma_i,U_i) \subseteq X(\kkk). \]
        The subset of form \(X_\kk(\{\sigma_i,U_i\}_{i=1}^n)\) is called a \emph{basic adelic open subset} of \(X(\kkk)\).
    \end{definition}

    \begin{remark}\label{rmk:motivation_of_adelic_topology}
        
    \end{remark}

    \begin{lemma}\label{lem:basic_fact_of_basic_adelic_subset}
        We have the following basic facts of basic adelic subsets:
        \begin{enumerate}
            \item let \(\kk'\) be a finite extension of \(\kk\) and \(X_{\kk'}(\{\sigma_i',U_i'\}_{i=1}^m)\) be a basic adelic subset of \(X(\kkk)\) defined by \(\kk'\);
            \item let \(f: Y \to X\) be a morphism of varieties over \(\kkk\) and \(X_\kk(\{\sigma_i,U_i\}_{i=1}^n)\) be a basic adelic subset of \(X(\kkk)\) defined by \(\kk\), 
                then \(f^{-1}(X_\kk(\{\sigma_i,U_i\}_{i=1}^n)) = Y_\kk(\{\sigma_i,f^{-1}(U_i)\}_{i=1}^n)\);
            \item let \(\tau \in \Gal(\kkk/\kk)\) and \(X_{\kk'}(\{\sigma_i',U_i'\}_{i=1}^m)\) be a basic adelic subset of \(X(\kkk)\) defined by \(\kk'\), then \(\tau(X_{\kk'}(\{\sigma_i',U_i'\}_{i=1}^m)) = X_{\kk'}(\{\tau\sigma_i',\tau(U_i')\}_{i=1}^m)\);
        \end{enumerate}
        \Yang{}
    \end{lemma}


    \begin{definition}\label{def:general_adelic_subset}
        A \emph{general adelic subset} of \(X(\kkk)\) is defined to a subset of the form \(\pi(B)\) with \(\pi: Y \to X\) a flat morphism of varieties and \(B\) a basic adelic open subset of \(Y(\kkk)\). 
    \end{definition}

    \begin{remark}\label{rmk:general_adelic_subset_and_define_field}
        To define a general adelic subset of \(X(\kkk)\), there are two fields involved: the field \(\kk\) over which the basic adelic subset is defined, 
        and the field \(\bfl\) over which the morphism \(\pi:Y \to X\) is defined. 

        If we fix a defined field \(\kk_0\) of \(X\), by \cite[Proposition 3.15 (ii) and (v)]{Xie25ZDOsurfaces}, we can always choose \(\bfl = \kk_0\) and \(\kk\) is a finite extension of \(\kk_0\). 
    \end{remark}

    % \begin{remark}\label{rmk:irreducible_variety_and_finite_union_of_general_adelic_subset}
    %     Note that our definition of general adelic subset is slightly different from the one in \cite{Xie25ZDOsurfaces}, where the author does not require that the variety is irreducible.
    %     % Note that on \cite{Xie25ZDOsurfaces}, the author does not require that varieties are irreducible, 
    %     if so, the finite union of general adelic subsets will be general adelic subsets in this sense.
    %     \Yang{To be revised.}
    % \end{remark}

    \begin{lemma}\label{prop:finite_union_and_intersection_of_general_adelic_subset}
        The finite union and intersection of general adelic subsets are still general adelic subsets.
    \end{lemma}
    \begin{proof}
        Let \(B_1,B_2\) be two general adelic subsets of \(X(\kkk)\) defined by \(\pi_i: Y_i \to X\) and basic adelic subsets \(A_i \subseteq Y_i(\kkk)\) for \(i=1,2\).
        Let \(Y = Y_1 \times_X Y_2\) and \(\pi: Y \to X\) be the natural morphism.
        Since we work over a field of characteristic zero, \(Y\) is reduced.
        Hence \(\pi: Y \to X\) is a flat morphism between varieties.
        Denote by \(C_i = p_i^{-1}(A_i)\) with \(p_i: Y \to Y_i\) the natural projection for \(i=1,2\).
        
        \Yang{To be added.}
    \end{proof}


    By \cref{prop:finite_union_and_intersection_of_general_adelic_subset}, the general adelic subsets form a basis of topology on \(X(\kkk)\).
    Hence we have the following definition.

    \begin{definition}\label{def:adelit_topology}
        The \emph{adelic topology} on \(X(\kkk)\) is defined to be the topology generated by all general adelic subsets of \(X(\kkk)\),
        i.e., an adelic open subset is an arbitrary union of general adelic subsets.
    \end{definition}

    \begin{proposition}\label{prop:basic_properties_of_adelic_topology}
        We have the following properties of adelic topology:
        \begin{enumerate}
            \item adelic topology is finer than Zariski topology;
            \item morphisms of varieties are continuous with respect to adelic topology;
            \item flat morphisms of varieties are open with respect to adelic topology.
        \end{enumerate}
    \end{proposition}
    \begin{proof}
        \Yang{To be added.}
    \end{proof}

    \begin{proposition}\label{prop:galois_action_is_continuous_wrt_adelic_and_profinite_topo}
        Let \(\kk\) be a defined field of \(X\).
        The action of \(\Gal(\kkk/\kk)\) on \(X(\kkk)\), namely \((\sigma,x) \mapsto \sigma(x)\), is continuous with respect to the adelic topology on \(X(\kkk)\) and the profinite topology on \(\Gal(\kkk/\kk)\).
    \end{proposition}
    \begin{proof}
        \Yang{To be added.}
    \end{proof}

    Recall that on \(\bbP^1_\bbQ\), the Artin-Whaples Approximation Theorem says that for any finite collection of places \(v_1,\ldots,v_n\) of \(\bbQ\) and any open subsets \(U_i \subseteq \bbP^1(\bbQ_{v_i})\), 
    the intersection \(\bigcap_{i=1}^n (U_i\cap \bbP^1(\bbQ))\) is non-empty.
    The following lemma is a generalization and it is the motivation of the definition of adelic subsets.

    \begin{theorem}\label{thm:intersection_of_adelic_open_subset}
        Adelic topology preserves the irreducibility of varieties.
        Explicitly, on a variety, the intersection of any finite collection of non-empty adelic open subsets is non-empty.
    \end{theorem}
    \begin{proof}
        \Yang{To be added.}
    \end{proof}

    \begin{lemma}[{ref. \cite[Proposition 3.9]{Xie25ZDOsurfaces}, cf. \cite[Theorem 1.2]{ManZan14:AWapproxOnVarieties}}]\label{lem:intersection_of_basic_adelic_subset_is_nonempty}
        For any finite collection of basic adelic open subsets \(X_\kk(\sigma_i,U_i)\), the intersection \(X_\kk(\{\sigma_i,U_i\}_{i=1}^n)\) is non-empty.
        % \[ \bigcap_{i=1}^n X_\kk(\sigma_i,U_i) \]
        % is non-empty.
    \end{lemma}
    \begin{slogan}
        On a variety, the small open ball in one place will be dense with respect to other places.
    \end{slogan}
    \begin{proof}
        \Yang{To be added.}
    \end{proof}


% \subsection{Adelic topology}

\subsection{Examples and relation to dynamics}



    \Yang{Is \(\bbA^1(\bbQ)\) adelic closed in \(\bbA^1(\overline{\bbQ})\) with adelic topology? If so, why?}

    \Yang{Describe all adelic closed subset in \(\bbA^1(\overline{\bbQ})\).}

