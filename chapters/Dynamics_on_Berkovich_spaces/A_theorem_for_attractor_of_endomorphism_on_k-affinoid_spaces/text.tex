\section{A theorem for attractor on k-affinoid spaces}

    In this section, we copy \cite[Appendix A]{Xie25ZDOsurfaces}.

    Fix an algebraically closed non-Archimedean complete field \(\kkk\) of characteristic zero and with non-trivial valuation, typically \(\kkk = \bbC_p\).

    \paragraph{Setup} Let \(A\) be a strictly and reduced \(\kkk\)-affinoid algebra, and \(X = \scrM(A)\) be its Berkovich spectrum.
    Let \(f: X \to X\) be a finite morphism of Berkovich spaces over \(\kkk\). 
    We further assume that there exists a subvariety \(\tilde{Z} \subseteq \tilde{X}\) such that \(\tilde{f}(\tilde{X}) = \tilde{Z}\) and \(\tilde{f}|_{\tilde{Z}}\) is an automorphism of \(\tilde{Z}\).
    

    \begin{definition}\label{def:f_spectral_radius_and_quasinilpotent_ideal}
        Let \(\rho_A\) be the spectral radius on \(A\). 
        For any \(g \in A\), we define
        \[ \rho_f(g) \coloneqq \lim_{n \to \infty} \rho_A\left( (f^*)^n(g) \right) \]
        where \(f^*: A \to A\) is the induced endomorphism on \(A\).
        Denote by
        \[ \Qnil_f \coloneqq \{ g \in A \mid \rho_f(g) = 0 \}. \]
    \end{definition}

    \begin{proposition}\label{prop:uniformly_converge_on_quinilpotent_ideal}
        There exists a constant \(0 < c < 1\) and \(m \geq 1\) such that for any \(g \in \Qnil_f\), we have
        \[ \rho_A\left( (f^*)^{m}(g) \right) \leq c \rho_A(g). \]
    \end{proposition}

    \begin{proposition}\label{prop:Y_is_Z_after_reduction}
        Set \(Y = V(\Qnil_f) \subseteq X\) be the closed analytic subspace defined by the ideal \(\Qnil_f\).
        We have \(\tilde{Y} = \tilde{Z}\).
        \Yang{To be checked}
    \end{proposition}
    
    The main result is the following theorem.

    \begin{theorem}[{ref. \cite[Theorem 8.3]{Xie25ZDOsurfaces}}]\label{thm:attractor_is_Zariski_closed_and_endomorphism_descends}
        We have 
        \begin{enumerate}
            % \item \(\widetilde{\Qnil_f} = I_{\tilde{Z}}\), the ideal of \(\tilde{Z}\) in \(\tilde{X}\);
            \item \(f|_{Y}: Y \to Y\) is an automorphism;
            \item there exists a unique \(\psi: X \to Y\) making \(Y \injmap X\) a section of \(\psi\), 
                such that \(f|_{Y} \circ \psi = \psi \circ f\); 
        \end{enumerate}
        Moreover, there exists \(C>0\), \(0 < \beta < 1\) such that for any \(x \in X\) and any \(h \in A\), we have
        \[ |h(f^n(\psi(x))) - h(f^n(x))| \leq C \beta^n \rho_A(h), \quad \forall n \geq 0. \]
        \Yang{To be revised.}
    \end{theorem}

\subsection{Zariski dense orbit in dimension two}

    \paragraph{Zariski dense orbit in dimension two}
    Let \(X = \scrM(\kkk{x,y})\) be the closed unit polydisc of dimension two over \(\kkk\).
    Suppose that \(f: X \to X\) satisfies 
    \[ \tilde{f}: (x,y) \mapsto (a x + b, 0). \]
    In this case, the attractor \(Y\) is the line defined by \(y = 0\), and \(f|_{Y}\) is an automorphism of \(Y\).\Yang{To be revised. }

    \begin{proposition}\label{prop:local_ZDO_of_surface}
        Suppose that \(f|_{Y}\) is not of finite order and \(f^{-1}(Y) \neq X\).
        Then there exists an affinoid subdomain \(U \subseteq X\) such that for every point \(x \in U\), the orbit \(\{ f^n(x) \}_{n \geq 0}\) is Zariski dense in \(X\).
        \Yang{}
    \end{proposition}